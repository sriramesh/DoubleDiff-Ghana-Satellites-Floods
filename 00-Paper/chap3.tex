\chapter{Literature Review}

Over the last 25 to 30 years, the explosive growth in computing power, data science, and technological advancement has created a market for data science products aimed specifically at transforming the decisions made by government policymakers everyday. These data science products come in one of many forms, from online dashboards, to interactive maps, to mobile applications. New trends in the public sector around ‘evidence-based’ or ‘data-driven’ decision making have expanded the market for data science products tenfold. In today’s world, policymakers have available to them a wide marketplace of data science products that claim to optimize, augment, and improve the efficacy, accuracy, efficiency, and speed of their decisions. However investing in these products comes at significant cost to governments; data science products, like all technology, are extremely expensive, usually require multi-year commitments from governments to install and roll out, and usually require significant cultural shifts in government agencies to embrace quantitative, data-driven modes of thinking.\\

Satellite-based flood information represents one example of a data science product that claims to optimize government decision making around emergency response. However, there is a dearth of literature on the efficacy of satellite-based flood information systems in actually improving observable development outcomes. The section outlines the two key pieces of literature that exist in this field, at the intersection of applied economics and remote sensing.\\

The first is a March 2018 paper by Oddo et. al.\cite{oddo2018socioeconomic} that provides an academic basis for a satellite-based flood information system developed for Ghana. The study integrates a socioeconomic damage assessment model with a near real-time flood remote sensing and decision support tool (NASA’s Project Mekong). Using the 2011 Southeast Asian flood as a case study, this tool is used to successfully derive flood-affected population, infrastructure and land cover estimates in the aftermath of this flood in the Mekong River Basin. Results of this study suggest that rapid initial estimates of flood impacts can provide valuable information to governments, international agencies, and disaster responders in the wake of extreme flood events. The second is a September 2019 study by Oddo et. al.\cite{oddo2019value} that points to the social and economic value of investing in satellite-based flood information systems. This study simulates a hypothetical flood event in Thailand, models vehicle routes and uses a value of information metric to quantify the social value and economic benefit of implementing a near-real-time flood impact system compared to ‘baseline routing strategies.’ Specifically, the study finds that the application of near real-time Earth observations can improve the response time and reduce potential encounters with flood hazards when compared with baseline routing strategies. Results indicate a potential significant economic benefit (i.e. millions of dollars) from applying near real-time Earth observations for improved flood disaster response and management.\\

In sum, both of these studies show promise that satellite-based flood information generates positive returns for policymakers. However, neither of these studies look at the relationship between satellites and food-based aid, and neither look at causation with respect to the impact of satellites on development outcomes. In terms of the relationship between satellites and development outcomes, the closest related work is the ongoing research study by Josh Blumenstock at UC Berkeley that examines the use of satellite imagery in optimizing aid targeting (i.e. cash-based transfers) in Togo\cite{blumenstocktogo}. This study however also does not examine the impact of satellites on improving food security outcomes, it only looks at using satellites to optimize cash-based transfers.\\

All in all, generating quasi-experimental evidence around the efficacy of satellite-based flood information platforms in improving community-level development outcomes would add valuable context to the conclusions derived from these prior studies, which do not use experimental or quasi-experimental methods. Understanding the impact of satellite-based flood information in development outcomes would provide insight into the ability of this technology to support climate resilience amongst flood-prone communities, and build the capacity of government policymakers to in turn strengthen the long-term flood preparedness of these communities. 
