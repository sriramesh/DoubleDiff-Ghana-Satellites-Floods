\chapter{Study Hypothesis}

The study hypothesis is that the FIS reduces food insecurity in high flood-risk Ghanaian districts relative to low-flood risk Ghanaian districts.\footnote{All of the information in this section comes from more than 15 semi-structured interviews, and internal meetings held with senior NADMO disaster management officials between May and November 2020.}\\

Taking a closer look at the nature of the FIS as a program, we see why this hypothesis makes sense. One of the main purposes of the FIS is to provide NADMO with the information it needs to optimize aid targeting decisions. These decisions can be described as the type, amount, and location of aid to provide after a flood has taken place. Next, one of the main forms of aid that NADMO provides is food aid. NADMO officers are interested in the appropriate placement of food aid in flood-prone regions. After flooding has taken place, food aid would be more readily available to those that need it most. This might have the effect of reducing the recovery time of these communities after flooding, according to NADMO officials.\\

As an additional point of evidence around the significance of food aid, NADMO’s partnership with UN WFP is what enables NADMO to use the FIS. WFP Ghana has provided NADMO with the funds to purchase the FIS via contract with Cloud to Street for the last three years. In sum, we see that a measure of food security at the household or district level would be an ideal outcome variable to examine programmatic impact.\\

One would also expect the benefits of the FIS to be most prevalent in the districts of Ghana that are at highest flood risk relative to those that are not at flood risk. This is because by leveraging satellites, the FIS provides a flood preparedness platform that promises to enable NADMO with the information they need to reduce the harm caused by flooding in the districts that are routinely hit the hardest. The flood frequency maps and near-real time flood damage estimates contained in the FIS are meant for NADMO to use in order to optimize its aid targeting activities. Anecdotally, NADMO officials report that they plan to use the FIS to determine the appropriate type and quantity of both food and cash-based aid for each district affected by flooding.

