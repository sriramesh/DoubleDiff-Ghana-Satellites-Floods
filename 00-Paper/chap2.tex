\chapter{Introduction}

Across the African continent, not only is severe flooding becoming more frequent during the rainy season because of climate change, poorer individuals are overexposed to flooding relative to those who are not poor.\cite{douglas2008unjust} This poverty-climate nexus is growing in scale; a growing number of people experiencing poverty live in floodplains, largely due to their limited alternatives in terms of housing\cite{abubakari2019cities} (i.e. people with limited income who cannot afford housing in non flood-prone regions tend to settle within flood-prone areas, which tend to be cheaper).\cite{mensah2020causes}\\

In Ghana, frequent, catastrophic flooding affects millions of people every year. Recurring flood events in Accra, Kumasi, Tamale, Sekondi-Takoradi, Eastern and Volta regions claims hundreds of lives and destroys valuable assets worth thousands of Ghana cedis yearly.\cite{abubakari2019cities} The central government of Ghana, as a result, is incurring larger annual costs in post-flood disaster relief due to the increased frequency and severity of natural disasters nationwide.\\

A critical component of preparing for, responding to, and recovering from flood emergencies is information management.\footnote{The information in the rest of this section comes from more than 15 semi-structured interviews, and internal meetings held with senior NADMO disaster management officials between May and November 2020.} In order to effectively coordinate the logistics of a flood preparedness, response or recovery operation, first responders with the Ghanaian central government, as well as humanitarian relief organizations such as the IFRC and the UN WFP, require accurate, timely and disaggregated data on flooding. Such accurate, timely and disaggregated data includes flood damage estimates. Flood damage estimates include the number of flood-struck civilians, the length of flood-destroyed roads, and the area of flood-affected cropland in any given region. It also includes geospatial and climatological information, such as flood extents, precipitation data and rainfall levels.\\

Information on flood damage is required to quickly and accurately target humanitarian assistance in the event of a flood. The aid targeting process involves at least three unique tasks: 

\begin{itemize}
    \item disbursing the right type of aid,
    \item disbursing the right amount of aid, and
    \item identifying the right individuals to give aid to.
\end{itemize}

Optimizing these three aspects of aid targeting requires timely and accurate flood damage estimates across Ghana. Within the flood response ecosystem in Ghana, the status quo in terms of obtaining such information is that first responders primarily rely on sources of ‘ground truth’. The main sources of ground truth are reports on flooding in the media, civilian hotline information, and need assessments in which teams of first responders visit flood-affected areas to collect damage estimates. By relying primarily on ground truth, first responders in Ghana are often unable to respond to flood events fast enough.\\

One promising source of flood information that could represent an improvement upon this status quo is satellite imagery. Using remote sensing technology, one can leverage the imagery provided by public satellites that circle the earth to map floods, and thus better understand where flooding is taking place across the world. By overlaying flood maps with geospatial data on key assets - such as the locations of buildings, households and roads - one then has a powerful data-driven tool to improve policy-making around flood preparedness, response, and recovery in flood-prone communities. This satellite-based flood information is a critical component of the Government of Ghana’s plan for improving the country’s long-term flood preparedness, and building its overall resilience to climate change.\\

Satellite-based flood information has its advantages and disadvantages. The key advantage is that it enables first responders to identify flooded regions of the country in near-real time, and may provide more accurate estimates of flood-caused damage in quicker amounts of time relative to ground truth sources. However, the disadvantages of satellite-based flood information include that it does not provide insight into the extent or the nature of the damage. In pilot studies of satellite-based flood information systems, governments often opt to send in first responders to flooded regions in order to verify the accuracy of the satellite-based flood information they might receive from external data providers.\\

The next section outlines gaps in the  literature on the efficacy of satellite-based flood information in improving development outcomes.
