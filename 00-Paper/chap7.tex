\chapter{External and Internal Validity}

The following discusses the methods used to establish external and internal validity in the study.\\

\section{External Validity}

Impact evaluations ideally employ both random selection and random assignment to establish external validity and internal validity, respectively. Because it is usually not possible to include the entire target population in the evaluation sample, random selection is often used to draw an evaluation sample from the target population.\\

In this study, however, it was possible to include the entire target population in the evaluation sample thanks to the rich repository of remotely-sensed flood information available for the analyses. Both the target population and the evaluation sample for this study is all 216 districts of Ghana. This means that inferences from the impact evaluation would apply directly to the target population of all districts in Ghana so long as internal validity is established.

\section{Internal Validity}

Internal validity in a study that involves causal inference such as this one is best ensured by using random assignment. This means that each sampling unit in the evaluation sample is randomly assigned to one of two groups: the treatment group, which receives the program, and the control group, which does not receive the program. Random assignment is the gold standard of causal inference; when randomization is successful, it ensures that observed differences between the treatment and control groups are an unbiased estimate of the average treatment effect within the evaluation sample.\\

A randomized evaluation could not be done for the program in question because the program could not be randomly assigned to the 216 Ghanaian districts in the evaluation sample. Specifically, the FIS was housed at the NADMO headquarters in Accra, and NADMO officials in Accra used the FIS to coordinate district-level aid distribution efforts on a periodic basis.\\

Where random assignment of the program is not feasible, one might look for a natural experiment instead.  One option is to use a double difference evaluation to draw causal inferences if the identifying assumption of parallel trends is satisfied. In a double difference evaluation, one compares average values of a given outcome variable of interest between a group that receives the treatment, and the group that does both before and after the introduction of the program. The double difference design accounts for the ‘before/after’ comparison and the ‘with program/without program’ comparison. The before/after and with/without scenarios for this study are summarized in Table 7.1.\\

\begin{table}
\centering
\begin{tabular}{|c|c|}
\hline
\textbf{Study design component} & \textbf{Description}\\
\hline
Before\rule{0pt}{4ex} & Before August 2018 \\
After\rule{0pt}{4ex} & After December 2020 \\
With\rule{0pt}{4ex} & Flood-prone districts of Ghana \\
Without\rule{0pt}{4ex} & Non flood-prone districts of Ghana \\
\hline
\end{tabular}
\caption{Study design components}
\end{table}

As shown in Table 7.1, the before-after comparison in this study is the time prior to August 2018 and the time after December 2020 respectively. The with-without comparison in this study is done using flood-prone districts of Ghana as the ‘with’ group and the non flood-prone districts of Ghana as the ‘without’ group.\\

The next section outlines all components of this study’s double difference evaluation.
