\chapter{Program Description}

This section provides an overview of the program and anecdotal, out-of-country evidence of the value of the program based on anecdotal evidence provided by NADMO.\footnote{All of the information in this section comes from more than 15 semi-structured interviews, and internal meetings held with senior NADMO disaster management officials between May and November 2020.}

\section{Overview}

The program in question in this study is a satellite-based Flood Information System or FIS, which was provided to NADMO, the Government of Ghana’s foremost agency for disaster risk management that is housed in the Ministry of Interior. Private data providers with expertise in the discipline of aggregating data from satellites, also known as remote sensing, have specialized in the development and sale of FIS platforms particularly for emergency response in LMICs. In August 2018, one such data provider, Cloud to Street, developed and implemented a FIS to augment NADMO's capabilities around flood preparedness, response, and recovery. For the last three years, Cloud to Street has provided NADMO with the FIS platform in an effort to optimize these three processes, at the request of the central government.\\

The FIS that Cloud to Street implemented for NADMO in 2018 is currently the Government of Ghana’s largest repository of flood information to-date. The FIS leverages public satellites to map floodplains and near-real time flood events. The platform itself has two main components:
\begin{enumerate}
  \item A 35-year annualized historical record of flood frequency data.
  \item Weekly, near-real time earth observations of ongoing flooding during Ghana’s rainy season (August to December of each year).
\end{enumerate}

As part of the near-real time monitoring service, the FIS has provided NADMO disaster management officials with quantifiable flood damage estimates as flooding is taking place. These estimates include the total number of people affected by flooding, the total area of flooded cropland, and the total number of schools affected by flooding in each district of Ghana each week. This information comes by way of an interactive dashboard, interactive maps, and downloadable files in machine-readable formats. Key flood alerts are provided to district-level NADMO officers via Whatsapp as well.\\

Since the program’s inception in August 2018, NADMO officials have used it to acquire near-real-time flood damage estimates to target humanitarian aid both faster, and more accurately. Anecdotally, NADMO officials report that they hope to use the FIS to augment its capabilities around anticipating floods prior to the flood striking. Achieving these objectives will enable NADMO to reduce the agency’s overall response time to flood events, and in turn, reduce the recovery time it takes flood-struck communities to bounce back. Response measures might include the time taken to preposition and disburse aid post-flooding, and recovery measures might include the time taken for individuals to recover lost wealth, and for farmers to resume farming activities.\\

The FIS has been operational in Ghana under the management of NADMO for 3 years: 2018, 2019, and 2020. In 2018, the FIS was launched as a pilot program for NADMO in collaboration with the African Risk Capacity, which is a division of the African Union. In 2019, NADMO launched a second pilot to further test the FIS capabilities. The first two pilot years were used to validate FIS information against ground truth estimates to evaluate its accuracy and implement functional improvements. In 2020, NADMO entered its first full, non-pilot year of FIS operations. That year, NADMO held a few agency-wide trainings on how to use the FIS to inform flood response operations. FIS information was purportedly disseminated to NADMO district officers to support district-level flood preparedness and response activities throughout the 2020 rainy season. With the end of the most recent rainy season in December 2020, NADMO concluded its use of the FIS. The agency plans to renew usage of the FIS for the 2021 rainy season.\\

The program start and end dates are summarized as follows:
\begin{itemize}
  \item Program start date: August 2018
  \item Program end date: December 2020
\end{itemize}

\section{Out-of-country anecdotal evidence of program value}

From May to November 2020, emergency management officials in the Republic of Congo were also interviewed to collect qualitative, anecdotal evidence on the value provided by a satellite-based FIS that their government had purchased in 2018. The first responders in the Congo reported that they were able to use the FIS to optimize aid targeting operations, such as where to target cash-based transfers. Implementing cash transfers requires a district to have functioning markets. The FIS provided the first responders with the location of functioning and flood-damaged markets, which they used to disburse cash-based and food-based assistance, respectively. According to these interviewees, the FIS information enabled these first responders to reach over 180,000 flood-affected Congolese civilians over the 9 months of the program’s operation.\\

In sum, the program description illustrates a potentially positive return on investment in the FIS. By using the FIS, NADMO officials might reduce (and thereby, improve) their agency-wide 'response time' to flooding across the country during the 2020 rainy season. If true, this in turn might engender reduced (and improved) bounce-back time for flood-prone districts in Ghana, where 'bounce back' might be measured in terms of food security.







